\chapter{Conclusions and Outlook} \label{chap:Conclusion}

The existing classification of \gls{IAV} is solely based on the surface proteins and gives, therefore, no real insight on the differences of the other segments. While many publications exist, involving evolutionary research on these, as well as, the other segments, that propose subgroups or clusters, most are based on alignments for comparison \autocite{suarez_evolution_2000, nelson_multiple_2008, whooiefao_h5n1_evolution_working_group_continued_2012}. Present day alignment methods offer accurate insight of sequence relation. Nevertheless, downsides of the use of \glspl{MSA} for the clustering of \gls{IAV} is the high number of available sequences and, thus, the necessity of high nearly unfeasible computational power, when involving all of them. Even with the hardware available a threshold for usability at a given number of sequences still exist or a drop of alignment quality is unavoidable. The usage of precalculation on the non-reduced $k$-mer frequency vectors mentioned in this project is another method aside from \gls{MSA} involving no dimension reduction methods. The exhibited results are, indeed, of mostly equal quality to the \gls{MSA} as shown in \autoref{sec:Comparison_Clustering} and drawing a similar clear line between subtypes. Still, similar downsides exist also, making scaling to the enormous number of existing \gls{IAV} sequences nearly impossible and render both methods mostly usable on smaller subsets.

\vspace{1em}

This project proposed a method scale-able to a much higher degree for clustering all segments of \gls{IAV}, that is usable on multi-core computers with around 32Gb of RAM available in less than two hours. The fast execution time of the method makes it usable to even cluster novel sequenced \gls{IAV} genomes with the existing ones to find associated well-known strains in short time. The dimension reduction with \texttt{PCA} proved to preserve the necessary amount of information for robust clustering and with the Kneedle Algorithm a solid threshold was defined. The present day version of the method already produced self contained clusters, with a higher order mostly in line with the current subtypes classification. Furthermore, increasing the amount of informations by reasonable subdivision in smaller subordinated groups. Still, there are options to fathom in the future to possibly produce even better results. 

\vspace{1em}

Including measurement possibilities for evolutionary distances by slight changes in the vector creation process as described in \autoref{fig:trans} could possibly improve the accuracy in comparison to \glspl{MSA}. %Furthermore, since the Kneedle Algorithm knee point selection is based on the definition of a search area a more robust method for the selection of the $\varepsilon$ value could be used. While the \gls{DBCV} as rating score did not produce the desired results in the exploration of appropriate $\varepsilon$ values, the Rand index also used in \textcite{viehweger_addressing_2019} might perform better by using ground knowledge available for segment 4 and 6. 
While \texttt{PCA} as tool for dimension reduction performed sufficient enough, the usage of a methods offering better low dimension representations and higher information preservation of the vectors could be beneficial. Therefore, repeating the comparisons to \texttt{PCA} with \texttt{t-SNE} instead of \texttt{UMAP} might be rewarding. Also all clusters and especially the high difference in the cluster sizes also have to be examined for possible clustering errors or rare mutations with possibly yet unknown purpose. 

\vspace{1em}

Still, the results point in the direction of a bioinformatical \gls{IAV} classification with more subdivisions as the known subtypes classification can offer. Renewing the classification in a similar way and, thereby, including more subtle differences might considerably improve future large scale  \textit{in silico} secondary structure analyses. With better and more self contained subgroups of \gls{IAV} searching for conserved structures would make a step ahead and could offer new insights of the \gls{IAV}.

%schau nochmal wegen den Enten black gull und so wegen evol ketten, so abschließend zum intro, blabla also evolutionary chains could be blabla falls es so ist

%Since exhaustion of the available time frame of this project, no further analysis were made, posterior to method and new classification proposal and the similarity matrix calculations as backup. Future research is thereby needed to further establish the classification by possible relations to specific strains or outbreaks of \gls{IAV}.

%precalc ground truth schön und gut aber nicht machbar
%Skalierbarkeit!!!!!
%laufzeit alignment (auch nicht immer perfekt)
%laufzeit kmer und speicher (am besten mit kmer)
%beides braucht TROTZDEM hdbscan (beides keine Cluster tools)
%kmer pca hdb skalierbar ohne exponentiell
%andere viren ohne probleme auch clusterbar (geringe anpassungen)
%new classfiication mit allen segmenten
%tool für jeden nutzbar
%globale classification basierend auf tool mit geringen anforderungen forschung influenza für jeden

%tool super schnell daher neuen strain sequencen und direkt in die pipeline schmeißen -> w gehört der hin nach 1,5 h

%repeat with tsne

%adujusted rand index instead of kneedle wahrscheinlich tausend mal besser