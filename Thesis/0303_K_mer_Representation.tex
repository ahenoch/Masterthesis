\section{The K-mer Representation Quality} \label{sec:K_mer_Representation}

Investigation on the anomalies resulted in two persistent clustering errors \autoref{fig:PCA_Cluster_Knee_4} \textbf{\textsf{B}} and \textbf{\textsf{D}}. To evaluate if the method is suitable for the clustering of \gls{IAV} possible error sources are discussed.

% \begin{figure}[!hbt]
%     \includegraphics[width=\dimexpr\textwidth-2\fboxsep-2\fboxrule,fbox]{PCA/Clustertree_Segment_4_H_Knee_Zoom.pdf}
%     \caption[H13/H16 Simple Clustering Example with \Acrshort{PCA}]{\textbf{H13/H16 Simple Clustering Example with \Acrshort{PCA}.} .}
%     \label{fig:PCA_Clusteree_Knee_Zoom}
% \end{figure}

% \begin{figure}[!hbt]
%     \includegraphics[width=\dimexpr\textwidth-2\fboxsep-2\fboxrule,fbox]{UMAP/Clustertree_Segment_4_H_Knee_Zoom.pdf}
%     \caption[H13/H16 Simple Clustering Example with \Acrshort{UMAP}]{\textbf{H13/H16 Simple Clustering Example with \Acrshort{UMAP}.} .}
%     \label{fig:UMAP_Clusteree_Knee_Zoom}
% \end{figure}

% \begin{figure}[!hbt]
%     \includegraphics[width=\dimexpr\textwidth-2\fboxsep-2\fboxrule,fbox]{UMAP/Guidetree_Segment_4_H_Focus.pdf}
%     \caption[H13/H16 Simple Clustering Example with \Acrshort{MSA}]{\textbf{H13/H16 Simple Clustering Example with \Acrshort{MSA}.} .}
%     \label{fig:Guidetree_Focus}
% \end{figure}

\begin{figure}[!hbt]
    \centering
    \includegraphics[width=\textwidth]{PCA/Precalculated_Segment_4_H_Cosine.pdf}
    \caption[H13/H16 Precalculated \Acrshort{UPGMA} Tree (cosine)]{\textbf{H13/H16 Precalculated \Acrshort{UPGMA} Tree (cosine).} .}
    \label{fig:Precalculated_Cosine}
\end{figure}

By building a \gls{UPGMA} tree on the non reduced segment 4 k-mer frequency vectors of the H13 and H16 subtypes the unbiased relation of sequences from these subtypes can be analyzed (\autoref{sec:MAFFT}). Also the fundamental use of k-mer frequencies can be validated or rejected. 

%frage is hier welche Methode die bessere representation ausstrahlt, sprich UMAP vgl. precalc ja/nein? dann PCA vgl precalc passt? ja/nein?