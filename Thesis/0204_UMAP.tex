\section{UMAP} \label{sec:UMAP}

\gls{UMAP} is a method for dimension reduction while aiming to preserve the global structure of the data \autocite{mcinnes_umap_2020}. In it's core it is similar to \gls{t-SNE} with better run time performance and better structure preservation in the lower dimension and no restrictions for embedding \autocite{mcinnes_umap_2020, maaten_visualizing_2008}. The default parameters of \gls{UMAP} used in the proposed tool for \gls{IAV} clustering mostly match the ones listed in the \href{https://umap-learn.readthedocs.io/en/latest/api.html}{API} under section \glqq \gls{UMAP} enhanced clustering\grqq{} \autoref{fig:Vectorization_Pipeline} \textsf{\textbf{c}}. The settings were used because they are proposed as settings to use prior to clustering with \gls{HDBSCAN}. Therefore \colorbox{backcolour}{min\_dist=0.0} setting was used to have better cluster separations and \colorbox{backcolour}{random\_state=42} to be able to more or less be able to reproduce the results. Regardless a higher dimension to embed the data in \colorbox{backcolour}{n\_components=30} was chosen to preserve more information, instead of visualize in two dimensions (\autoref{sec:PCA}). Based on the size of the sequences represented by $\mathbf{X}$ e.g. 56617 with segment 4, the \colorbox{backcolour}{n\_neighbors=100} setting was used to make \gls{UMAP} focus more on the global structure of the data. Also based on the input matrix size \colorbox{backcolour}{n\_epochs=200} setting was used automatically.

\autoref{eq:PCA_100} to \autoref{eq:UMAP_100} denote the use of the \gls{PCA} as shown in \autoref{fig:Vectorization_Pipeline} pathway \textsf{\textbf{2}} and the default setting for Jupyter results with the use of \gls{UMAP} \autocite{kluyver_jupyter_2016, mcinnes_umap_2020, pedregosa_scikit-learn_2011, jolliffe_principal_2016}.

\begin{empheq}{alignat = -1}
    &\mathbf{X}_{(100)} = \text{PCA}(\mathbf{X}, 100)\label{eq:PCA_100}
\end{empheq}

\begin{empheq}{alignat = -1}
    &\mathbf{Y} &&= \text{UMAP}(\mathbf{X}_{(100)}, n, d, \text{min\_dist}, \text{n-epochs})\label{eq:UMAP_d}\\
    &&&= \text{UMAP}(\mathbf{X}_{(100)}, 100, 30, 0.0, 200)\label{eq:UMAP_100}
\end{empheq}

The parameters used in this project with settings varying from the default are listed below. All available settings can be fount in the \href{https://umap-learn.readthedocs.io/en/latest/api.html}{API}.

\begin{leftbar}
    \textbf{umap.UMAP}
    \begin{nstabbing}
        \qquad\qquad\qquad\qquad\qquad\quad\=\kill

        n\_neighbors \> [number of neighbors (default: 15)]\\
        
        min\_dist \> [min. package distance of points (default: 0.1)]\\
        
        n\_components \> [dimension to embed in (default: 2)]\\
        
        n\_epochs \> [number of epochs for training (default: None)]\\
        
        metric \> [metric to use (default: 'euclidean')]
    \end{nstabbing}
\end{leftbar}

\gls{UMAP} was used posterior to dimension reduction with \gls{PCA}, by reason of it's similarity to \gls{t-SNE}. As declared in the \href{https://scikit-learn.org/stable/modules/generated/sklearn.manifold.TSNE.html}{API} of \gls{t-SNE}, the dimension should be reduced to a reasonable amount prior to execution to reduce noise. In the \href{https://umap-learn.readthedocs.io/en/latest/api.html}{API} of \gls{UMAP} in section \glqq What is the difference between PCA / UMAP / VAEs?\grqq{} a pipeline is proposed, to reduce from high dimension with \gls{PCA}, continue with reduction by \gls{UMAP} and cluster with \gls{HDBSCAN}. Furthermore \gls{UMAP} with $d=30$ posterior to \gls{PCA} with $k=100$ provided a comfortable balancing of computational effort of both methods, while preserve $\approx 85\%$ explained variance with \gls{PCA} \autocite{mcinnes_umap_2020}. Since the final dimensionality of the resulting matrix $\mathbf{Y}$ either way is $<50$ a reduction with \gls{PCA} to $>50$ could be used while still preserving the usability of Spanning Tree calculation with \gls{HDBSCAN} \autocite{mcinnes_hdbscan_2017}. However it is indeed also possible to use \gls{UMAP} without prior \gls{PCA}, since there is no limit for input date dimensionality \autocite{mcinnes_umap_2020}. 