\section{Differences in Dimension Reduction} \label{sec:Dimension_Reduction}

To investigate the processing behavior prior to the clustering and thereby find explanations for the mentioned errors, the small H13 and H16 subset of segment 4 k-mer frequencies, were reduced by \gls{PCA} and \gls{UMAP} to two components to give the opportunity for in detail visualization. Comparison to \gls{UMAP} was done although the method was already declared as not appropriate, to validate this statement again and also see the impact with different neighbor values. 

The target of the reduction prior to the \gls{HDBSCAN} clustering, was to find a representation of the data that is most suitable to be used for the clustering, by preserving the information with a lower complexity. As declared by \autoref{sec:K_mer_Representation} and \autoref{sec:Comparison_Clustering} the optimal representation of the vectors should make a clear difference between H13 and H16 is thereby used as the ground truth in the following (\autoref{fig:Precalculated_Cosine}).

The visualization of the reduction by \gls{PCA} is denoted as neighbors value -1 \autoref{fig:Reduction_Comparison}. It shows five different accumulations of points. Coloring of these points is based on the original clustering example in \autoref{fig:PCA_Cluster_Knee_4}. This is becoming apparent when focusing on the orange cluster 48 points containing H13 and H16 sequences. That way a fundamental distribution on the points of H13 and H16 can be reviewed as well. 
%
%When using the right side as possible indication for clustering, all the points and accumulations of points are very close to each other. Nonetheless a separation with a imagined clustering can be made very easy by building two clusters of the blue points, one of the red and green and three of the orange points. Still, all the points related to H13 would be merged with the orange ones of H16 before the orange H16 points would be merged with the green ones of H16. Thereby the difference between H13 and H16, the higher-ranking goal would be not accomplished because the orange points are so close to each other. 

The reduction with \gls{PCA} on the subset results in easy separable blue and red points accumularion. The distribution of the points is basically in line with the result shown in \autoref{fig:PCA_Cluster_Knee_4}, as the accumulations of red and blue colored points are well separated, building two clusters with the same sequences. The major difference however, is the distance between the accumulations of range points to each other as well as to the green ones. This would probably result in a imaginary clustering of one red, one blue, as well as two orange clusters of which one also contains the green points. It seems as if the distance of the green points and the H13 orange points is underestimated to an great extend. By reduction to two dimensions, the difference between cluster 46 and 45 in the left picture is preserved and would result as shown in \autoref{fig:PCA_Cluster_Knee_4}, while on the other hand building at least one cluster merging the subtypes. Cluster 47 and 48 are next to each other in \autoref{fig:PCA_Cluster_Knee_4} and would be linked on the next tree-node.

It appears as if the points colored green and orange are possibly quite similar, which is not the case as the \autoref{fig:Precalculated_Cosine} sub trees clearly show the wanted separation of H13 and H16 in cluster 48 as well as the distance to 47. Keeping the lower complexity and the easier reduction of the sequences in this example, through the smaller number of sequences in mind, the consequence of lowering the dimension by \gls{PCA} to two dimension seems to preserve most of the information related to the difference of cluster 45 and 46. The difference of the subtype separation inside 48 as well as the overall difference to 47 seems to be lost completely. Since the ground truth separation of \autoref{fig:Precalculated_Cosine} seems to be partially present in \autoref{fig:PCA_Cluster_Knee_4}, by at least separating 47 completely from 48, the higher number of dimension might be in direct connection to the correct separation of some part of H13 and H16. Therefore the number of components should be increased to the maximum of 50, that still preserves all functions of \gls{HDBSCAN} for spanning-tree building. 

Comparing these results to the use of \gls{UMAP} with different settings of the neighbors value, the impact of this parameter becomes clear. The higher the value, to more crowded the points. This explains the behaivior in \autoref{subfig:Normalisation_UMAP}. Since a neighbors value of 100 was used as standard, the values are overall crowded in groups of at least 100 points. The random subset for the graphic was reduced by the same setting. The small random sample with a high neighbors value resulted in a low number of overall distribution. The small subset was used with this high standard value to clarify this behavior. Due to the size of the dataset used for the project a high value for neighbors was used as described in \autoref{sec:UMAP}. The same value as well as 15 and 50 neighbors was used on the subset of H13 and H16 segment 4 sequences to visualize the difference. 

None of the settings rsults in a separation as good as with the sole use of \gls{PCA}. With the \gls{UMAP} standard neighbors value of 15 all the points are next to each other and it is difficult to separate them in to clusters. Furthermore H13 points would be merged with H16 points before merging with others from H13, thereby breaking the subtype division. Still this setting would separate the points somewhat like in \autoref{fig:PCA_Cluster_Knee_4} since no blue and yellow points are mixed. Setting the neighbors value to 50 results in a spreading of the blue points and mixing with little isles of yellow points. The color of the points are based on the clustering of \autoref{fig:PCA_Cluster_Knee_4} containing discussed clustering errors, thereby the coloring must not match. Nonetheless like with neighbors 10 the subtypes overlap making separation impossible. With a neighbors value of 100 separation in quite similar clusters as with \gls{PCA} or as shown in \autoref{fig:PCA_Cluster_Knee_4} would be possible. However subtype separation isn't possible either even when ignoring the green points that might be very sensible to the magnitude of preserved information.

%With normalization and without some information seem to be missing necessary to separate the orange points and the green ones. While on the left side the distance was underestimated to an extend making the orange H13 points and the green H16 points collide, the distance on the right side is overestimated, making the subtype distance of the H13 and H16 orange points to small. Since the separation between red and blue, as well as H13 orange and H16 orange is clearer, the method using normalization is still proved to be the better one, in the circumstances that the location of the green points is caused by the low dimension which is proved by \autoref{fig:PCA_Cluster_Knee_4} showing a seperation between 47 and 48 and the right method not producing any better results related to the green points. 

\begin{figure}[hbt]
    \centering
    \includegraphics[width=\textwidth]{PCA/Difference_Segment_4_H_metric_cosine.pdf}
    \caption[H13/H16 Component Reduction Comparison]{\textbf{H13/H16 Component Reduction Comparison.} .}
    \label{fig:Reduction_Comparison}
\end{figure}

Concluding the use of \gls{PCA} seems to give better results than comparable ones with \gls{UMAP}. Still there are challenges to overcome as could be seen with the position of th green points. Maybe increasing the information preserved by the \gls{PCA} would give clearer results. 

The objective of this project was to find high-quality representations of \gls{IAV} genomes for the purpose of clustering in a to my best knowledge never done extend. Therefore the usability of \gls{HDBSCAN} with good parameters was of higher importance than the use of \gls{UMAP} at all costs. In the results of thie project \gls{PCA} performed better than \gls{UMAP} in any case but only with the parameters used. Therefor it might be possible to find parameters for \gls{UMAP} not explored in this project to represent the genomes even better in a relatively low dimension. Due to the limited time for time for this project, not all parameters given in the \href{https://umap-learn.readthedocs.io/en/latest/api.html}{API} could be tested and the solely use of \gls{PCA} instead outperformed the tested ones. 