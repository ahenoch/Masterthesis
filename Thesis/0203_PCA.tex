\section{Principal Component Analysis} \label{sec:PCA}

\gls{PCA} was used to handle the complexity of the vectors by simplification with the least loss of information possible (\autoref{fig:Clustering_Pipeline} \textsf{\textbf{C}}) \autocite{pearson_liii_1901}.%\gls{PCA} is a statistical technique, used to find a new presentation of the dataset with a lower complexity by variance maximizing, uncorrelated variables, the \glspl{PC}, which are linear functions of the previous ones \autocite{jolliffe_principal_2016}. 

%For the \gls{PCA} of matrix $\mathbf{X}_{\text{L1}}$, the \textbf{decomposition.PCA} function from the \textbf{scikit-learn (sklearn)} package version 0.24.1, with \colorbox{backcolour}{n\_components=30} setting was used \autocite{pedregosa_scikit-learn_2011}.

%The matrix of $\mathbf{X}_{\text{L1}}$ for e.~g.~, segment 4 of \gls{IAV} is of size $56617 \times 4^7$ and 

30 components were extracted by the PCA \autocite{pedregosa_scikit-learn_2011}. Extraction of 30 components out of $4^7$ in total is $\approx 0.18\%$. The size limit of the PCA function for calculation with default setting \texttt{svd\_solver='auto'} is 500 different vectors with 500 components and at least 80\% of the components to extract. Since every maximum for standard settings was exceeded, \texttt{svd\_solver='randomized'} setting was used automatically \autocite{pedregosa_scikit-learn_2011}. Randomized truncated \gls{SVD} is performed in the PCA as described in \textcite{halko_finding_2010}.

% \autoref{eq:PCA_k} and \autoref{eq:PCA_30} illustrate the use of the \gls{PCA} by the method of \textcite{halko_finding_2010} to reduce the matrix $\mathbf{X}_{\text{L1}}$ with $j=4^7$ dimensions to $k=30$ dimensions \autocite{jolliffe_principal_2016, pedregosa_scikit-learn_2011}. This calculation is a representation of the approach used in \autoref{fig:Clustering_Pipeline} workflow \textsf{\textbf{1}} and the settings for the Jupyter results with the use of \gls{PCA} only \autocite{kluyver_jupyter_2016}.

% \begin{empheq}{alignat = -1}
%     &\mathbf{X}_{\text{PCA}} &&= \text{PCA}(\mathbf{X}_{\text{L1}}, \text{n\_components})\label{eq:PCA_k}\\
%     &&&= \text{PCA}(\mathbf{X}_{\text{L1}}, 30)\label{eq:PCA_30}
% \end{empheq}

% Let $\mathbf{S}$ be a square covariance matrix of size $i \times i$, calculated on the dataset matrix $\mathbf{X}_{\text{L1}}$ with $i$ $j$-dimensional vectors (size $i \times j)$. 

% \autocite{jolliffe_principal_2016}.

% \begin{empheq}{alignat = -1}
%     x^\ast_{i,j} = x_{i,j} - \bar{x}_j
% \end{empheq}

% \begin{empheq}{alignat = -1}
%     &\mathbf{X}_{\text{L1}}^\ast &&= \mathbf{U} \mathbf{L} \mathbf{A}^\top\\
%     &&&= \mathbf{Y}
% \end{empheq}

% \begin{empheq}{alignat = -1}
%     &\mathbf{X}_{\text{L1}}^{\ast^\top} \mathbf{X}_{\text{L1}}^\ast &&= (\mathbf{U} \mathbf{L} \mathbf{A}^\top)^\top (\mathbf{U} \mathbf{L} \mathbf{A}^\top)\\
%     &&&= \mathbf{A}\mathbf{L}\mathbf{U}^\top\mathbf{U} \mathbf{L} \mathbf{A}^\top\\
%     &&&= \mathbf{A}\mathbf{L}^2\mathbf{A}^\top 
% \end{empheq}

% \begin{empheq}{alignat = -1}
%     \mathbf{Y}_{30} = \mathbf{U}_{30} \mathbf{L}_{30} \mathbf{A}_{30}^\top
% \end{empheq}

% Important parameters used in this project, including settings varying from the default, are listed below. All available settings with explanation are listed in the
% \href{https://scikit-learn.org/stable/modules/generated/sklearn.decomposition.PCA.html}{API} \autocite{pedregosa_scikit-learn_2011}.

% \begin{empheq}{alignat = -1}
%     &\mathbf{X}_{\text{L1}} = \begin{bmatrix}x_{1,1} & x_{1,2} & x_{1,3} & \dots & x_{1,30}\\
%     x_{2,1} & x_{2,2} & x_{2,3} & \dots & x_{2,30}\\
%     x_{3,1} & x_{3,2} & x_{3,3} & \dots & x_{3,30}\\
%     \vdots & \vdots & \vdots & \ddots & \vdots\\
%     x_{i,1} & x_{i,2} & x_{i,3} & \dots & x_{i,30}
%     \end{bmatrix}\label{eq:full_matrix_2}
% \end{empheq}

% \begin{leftbar}
%     \textbf{sklearn.decomposition.PCA}
%     \begin{nstabbing}
%         \qquad\qquad\qquad\qquad\qquad\quad\=\kill

%         svd\_solver \> [Strategy used to solve the \gls{SVD} (default: auto)]\\
%         n\_components \> [Number of components to extract (default: None)]
%     \end{nstabbing}
% \end{leftbar}