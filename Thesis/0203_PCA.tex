\section{PCA} \label{sec:PCA}

To handle the complexity of the datasets generated in the project, as well as to simplify them with the least loss of information as possible \gls{PCA} was used. 

\autocite{jolliffe_principal_2016}
\autocite{pearson_liii_1901}

The parameters used in this project with settings varying from the default are listed below. All available settings can be fount in the
\href{https://scikit-learn.org/stable/modules/generated/sklearn.decomposition.PCA.html}{API} \autocite{pedregosa_scikit-learn_2011}

\begin{leftbar}
    \textbf{sklearn.decomposition.PCA}
    \begin{nstabbing}
        \qquad\qquad\qquad\qquad\qquad\quad\=\kill

        n\_components \> (default: None)
    \end{nstabbing}
\end{leftbar}

\begin{table}[!hbt]
    \centering
    \caption[Explained Variance by different PCA settings]{\textbf{Explained Variance by different PCA settings.}.}
    \label{tab:PCA_Dimension}
    \pgfplotstabletypeset[
        every head row/.style={
            before row={
                \toprule
            },
            after row={
                \midrule
            },
        },
        every last row/.style={
            after row={
                \bottomrule
            },
        },
        begin table=\begin{tabular*}{.5\textwidth},
        end table=\end{tabular*},
        columns={0,1},
        columns/0/.style={int detect, multicolumn names=l,column name=\textbf{\#Components}, column type=@{\extracolsep{\fill}\hspace{6pt}}r},
        columns/1/.style={multicolumn names=l,column name=\textbf{Explained Variance}, column type=r}
    ]
    {Graphics/PCA.csv}
\end{table}