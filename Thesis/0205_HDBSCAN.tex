\section{HDBSCAN} \label{sec:HDBSCAN}

\gls{HDBSCAN} is an clustering algorithm proposed by \textcite{campello_hierarchical_2015}. It is a novel version of \gls{DBSCAN} but executed with varying values of epsilon \autocite{hutchison_density-based_2013}. Thus not one specific threshold is used to define the clusters, but instead clusters of varying densities are extracted based on their stability over epsilon \autocite{mcinnes_hdbscan_2017}. The version of \gls{HDBSCAN} used in this project is the accelerated implementation of \textcite{mcinnes_accelerated_2017} (\autoref{fig:Vectorization_Pipeline} \textsf{\textbf{E}}).

\autocite{mcinnes_accelerated_2017}

\autocite{malzer_hybrid_2020}



The parameters used in this project with settings varying from the default are listed below. All available settings can be fount in the \href{https://hdbscan.readthedocs.io/en/latest/api.html}{API}.

\begin{leftbar}
    \textbf{hdbscan.HDBSCAN}
    \begin{nstabbing}
        \qquad\qquad\qquad\qquad\qquad\quad\=\kill

        min\_cluster\_size \> (default: 5)\\
        
        min\_samples \> (default: None)\\
        
        cluster\_selection\_epsilon \> (default: 0.0)\\
        
        gen\_min\_span\_tree \> (default: False)
        
        %alpha \> (default: 1.0)
    \end{nstabbing}
\end{leftbar}

%Clustering
%DBCV mehr als 50dims abkacken bla
%Metrik
%Hybrid selection
%Needle
%Linkage matrix