\chapter*{Zusammenfassung}

Wiederauftretende lokale Ausbrüche von hoch pathogenen \gls{IAV} Strängen, verweisen auf eine ständig presente Gefahr für die Menschheit, die in der Vergangenheit mehrfach globale Ausmaße mit hohen Todeszahlen zufolge hatte. Da kein verlässliches Medikament vorhanden ist, muss unausweichlich auf Impfungen zurückgegriffen werden, die verschieden hohe Immunisierung gewähren und jährliche Erneuerung erfordern. Eine Erweiterung der Kenntnisse über \gls{IAV}, ist daher unerlässlich, um eine bessere Vorbereitung auf mögliche zukünftige Pandemien zu ermöglichen. Hohe Evolutionsraten durch die drastischeren Mutationsmechanismen von \gls{IAV} und eine wenig Einsicht gewährenden Klassifizierung, verkomplizieren dabei allerdings die Gewinnung neuer und exakter Forschungsergebnisse. Diese Thesis dient daher der Herausarbeitung einer Pipeline, zur segmentweisen Klassifizierung aller sequenzierten Genome von \gls{IAV}. Anstelle von Alignments, nutzt diese Methode besser skalierbare $k$-mer Frequenz-Vektoren mit dem neuen hybriden Clustering Ansatz von \texttt{HDBSCAN}, der hierarchische und auf Dichte basierende Methoden vereint. Geeignete Parameter wurden mit Hilfe verschiedener Validierungs-Techniken ausgesucht und die Dimensionalität der genutzten Vektoren mit bekannten Tools verringert. Die Ergebnisse wurden im Detail verglichen, wodurch ein Workflow kreiert wurde, der eine neue Clustermethode, mit validierten Parametern und einer zuvor erfolgten effizienten Reduzierung der Dimensionen, verbindet.