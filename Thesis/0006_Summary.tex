\chapter*{Zusammenfassung}

Wiederauftretende lokale Ausbrüche von hoch pathogenen \gls{IAV} Strängen, verweisen auf eine ständig presente Gefahr für die gesamte Menschheit, die in der Vergangenheit mehrfach globale Ausmaße mit millionen Toten zufolge hatte. Da kein wirkliches Heilmittel vorhanden ist, muss unausweichlich auf Impfungen zurückgegriffen werden, die verschieden hohe immunisierung gewähren und jährlich ablaufen. Eine Erweiterung der Kenntnisse über \gls{IAV}, ist daher unerlässlich, um eine bessere Vorbereitung auf mögliche zukünftige Pandemien zu ermöglichen. Hohe Evolutionsraten durch die drastischeren Mutationsmechanismen von \gls{IAV}, mit einer veralteten, wenig Einsicht gewährenden Klassifizierung, kompliziert allerdings die Gewinnung neuer und exakter Forschungsergebnisse. Diese Thesis dient daher der Herausarbeitung einer Pipeline, die genutzt werden kann, um alle existierenden und zukünftig sequenzierten Genome von \gls{IAV} neu zu klassifizieren. Anstelle von Alignments, nutzt diese Methode besser skalierbare $k$-mer Frequenz-Vektoren mit dem neuen hybriden Clustering Ansatz von \texttt{HDBSCAN}, der hierarchische und auf Dichte basierende Methoden vereint. Am besten geeignete Parameter wurden mit Hilfe verschiedener Validierungs-Techniken sorgfältig ausgesucht. Um den Informationsgehalt der Vektoren, in einer Dimension die sinnvolles Clustern ermöglicht, so gut wie möglich zu erhalten, wurden verschiedene Tools getest. Zusammenfassend wurde ein Workflow kreiert, der eine neue Clustermethode, mit validierten Parametern und einer zuvor erfolgten Reduzierung der Dimension, die den höchste Menge an Informationen konserviert, verbindet.